\documentclass[fleqn]{article}
\usepackage[a4paper, total={6in, 10in}]{geometry}
\usepackage{graphicx}
\usepackage{amsmath}
\graphicspath{ {./images/} }

\title{MTRN4230 Project 1}
\author{Lachlan Scott | z5207471}
\date{July 10, 2022}

\begin{document}
\maketitle

\section{Provided Data}
\subsection{Defined positions:}
A and B are two sets of joint angle positions ($\theta_1$ to $\theta_6$) of a UR5e robot:
\begin{itemize} 
    \item Joint Variables/Position A: [ -90, -173, 132, 220, 0, 0] (degrees)
    \item Joint Variables/Position B: [-90, -60, 90, 0, 90, 0] (degrees)
\end{itemize}
\subsection{Denavit-Hartenburg Parameter Table of UR5e Robot:}

\begin{table}[!ht]
    \caption{DH Parameters of UR5e Robot}
    \begin{center}
    \begin{tabular}{||c | c | c | c | c||} 
     \hline
     Link & $\theta$ (rad) & a (m) & d (m) & $\alpha$ (rad)\\
     \hline\hline
     1 & 0 & 0 & 0.1625 & $\frac{\pi}{2}$ \\ [1ex] 
     \hline
     2 & 0 & -0.425 & 0 & 0 \\ [1ex] 
     \hline
     3 & 0 & -0.3922 & 0 & 0 \\ [1ex] 
     \hline
     4 & 0 & 0 & 0.1333 & $\frac{\pi}{2}$ \\ [1ex] 
     \hline
     5 & 0 & 0 & 0.0997 & $-\frac{\pi}{2}$ \\ [1ex] 
     \hline
     6 & 0 & 0 & 0.0996 & 0 \\ [1ex] 
     \hline
    \end{tabular}
    \end{center}
\end{table}

\section{Part A: Kinematic Modelling}
\subsection{Construction of a Model of the UR5e Robot}

\subsection{Calculation of Frame Transformation Matrices}
\subsubsection{Useful Definitions}
Translation matrix describing a translation from frame A to frame B:
\begin{equation}
\begin{split}
^{A}p_{B}=\begin{bmatrix}
^{A}x_{B}\\
^{A}y_{B}\\
^{A}z_{B}
\end{bmatrix}
\end{split}
\end{equation}
Rotation matrix describing a rotation from frame A to frame B:
\begin{equation}
\begin{split}
^{A}R_{B}=\begin{bmatrix}
x_b\cdot x_a & y_b\cdot x_a & z_b\cdot x_a\\
x_b\cdot y_a & y_b\cdot y_a & z_b\cdot y_a\\
x_b\cdot z_a & y_b\cdot z_a & z_b\cdot z_a
\end{bmatrix}
\end{split}
\end{equation}
Transformation between consecutive joints using DH convention:
\begin{equation}
\begin{split}
^{i-1}T_{i}&=R_{i-1}(\theta_i)\cdot Q_{i-1}(d_i)\cdot Q_i(a_i)\cdot R_i(\alpha_i)\\
&=\begin{bmatrix}
\cos(\theta_i) & -\sin(\theta_i)\cos(\alpha_i) & \sin(\theta_i)\sin(\alpha_i) & a_i\cos(\theta_i)\\
\sin(\theta_i) & \cos(\theta_i)\sin(\alpha_i) & -\cos(\theta_i)\sin(\alpha_i) & a_i\sin(\theta_i)\\
0 & \sin(\alpha_i) & \cos(\alpha_i) & d_i\\
0 & 0 & 0 & 1
\end{bmatrix}
\end{split}
\end{equation}
\subsubsection{Frame Transformation Matrices of Joint Position A}
Obtain the DH table of the robot at joint position A by converting specified joint angles from degrees to radians:
\begin{table}[!ht]
    \caption{DH Parameters of UR5e Robot at position A}
    \begin{center}
    \begin{tabular}{||c | c | c | c | c||} 
     \hline
     Link & $\theta$ (rad) & a (m) & d (m) & $\alpha$ (rad)\\
     \hline\hline
     1 & $-\frac{\pi}{2}$ & 0 & 0.1625 & $\frac{\pi}{2}$ \\ [1ex] 
     \hline
     2 & -3.019 & -0.425 & 0 & 0 \\ [1ex] 
     \hline
     3 & 2.304 & -0.3922 & 0 & 0 \\ [1ex] 
     \hline
     4 & 3.840 & 0 & 0.1333 & $\frac{\pi}{2}$ \\ [1ex] 
     \hline
     5 & 0 & 0 & 0.0997 & $-\frac{\pi}{2}$ \\ [1ex] 
     \hline
     6 & 0 & 0 & 0.0996 & 0 \\ [1ex] 
     \hline
    \end{tabular}
    \end{center}
\end{table}

Now, the consecutive joint frame transformations can be obtained by substitution of the above values: 
\begin{equation}
\begin{split}
^{0}T_{1}&=\begin{bmatrix}
\cos(-\frac{\pi}{2}) & -\sin(-\frac{\pi}{2})\cos(\frac{\pi}{2}) & \sin(-\frac{\pi}{2})\sin(\frac{\pi}{2}) & 0\cos(-\frac{\pi}{2})\\
\sin(-\frac{\pi}{2}) & \cos(-\frac{\pi}{2})\sin(\frac{\pi}{2}) & -\cos(-\frac{\pi}{2})\sin(\frac{\pi}{2}) & 0\sin(-\frac{\pi}{2})\\
0 & \sin(\frac{\pi}{2}) & \cos(\frac{\pi}{2}) & 0.1625\\
0 & 0 & 0 & 1
\end{bmatrix}\\
&=\begin{bmatrix}
0 & 0 & -1 & 0\\
-1 & 0 & 0 & 0\\
0 & 1 & 0 & 0.1625\\
0 & 0 & 0 & 1
\end{bmatrix}
\end{split}
\end{equation}

\begin{equation}
\begin{split}
^{1}T_{2}&=\begin{bmatrix}
-0.9925 & 0.1219 & 0 & 0.4218\\
-0.1219 & -0.9925 & 0 & 0.0518\\
0 & 0 & 1 & 0\\
0 & 0 & 0 & 1
\end{bmatrix}
\end{split}
\end{equation}

\begin{equation}
\begin{split}
^{2}T_{3}&=\begin{bmatrix}
-0.6691 & -0.7431 & 0 & 0.2624\\
0.7431 & -0.6691 & 0 & -0.2915\\
0 & 0 & 1 & 0\\
0 & 0 & 0 & 1
\end{bmatrix}
\end{split}
\end{equation}

\begin{equation}
\begin{split}
^{3}T_{4}&=\begin{bmatrix}
-0.7660 & 0 & -0.6428 & 0\\
-0.6428 & 0 & 0.7660 & 0\\
0 & 1 & 0 & 0.1333\\
0 & 0 & 0 & 1
\end{bmatrix}
\end{split}
\end{equation}

\begin{equation}
\begin{split}
^{4}T_{5}&=\begin{bmatrix}
1 & 0 & 0 & 0\\
0 & 0 & 1 & 0\\
0 & -1 & 0 & 0.0997\\
0 & 0 & 0 & 1
\end{bmatrix}
\end{split}
\end{equation}

\begin{equation}
\begin{split}
^{5}T_{6}&=\begin{bmatrix}
1 & 0 & 0 & 0\\
0 & 1 & 0 & 0\\
0 & 0 & 1 & 0.0996\\
0 & 0 & 0 & 1
\end{bmatrix}
\end{split}
\end{equation}

With the consecutive joint frame transformations, the transformations between each joint and the base frame are easily obtained:

\begin{equation}
\begin{split}
^{0}T_{2}&=^{0}T_{1}\cdot^{1}T_{2}
&=\begin{bmatrix}
0 & 0 & -1 & 0\\
0.9925 & -0.1219 & 0 & -0.4218\\
-0.1219 & 0.9925 & 0 & 0.2143\\
0 & 0 & 0 & 1
\end{bmatrix}
\end{split}
\end{equation}

\begin{equation}
\begin{split}
^{0}T_{3}&=^{0}T_{2}\cdot^{2}T_{3}
&=\begin{bmatrix}
0 & 0 & -1 & 0\\
-0.7547 & -0.6561 & 0 & -0.1258\\
-0.6561 & 0.7547 & 0 & 0.4716\\
0 & 0 & 0 & 1
\end{bmatrix}
\end{split}
\end{equation}

\begin{equation}
\begin{split}
^{0}T_{4}&=^{0}T_{3}\cdot^{3}T_{4}
&=\begin{bmatrix}
0 & -1 & 0 & -0.1333\\
0.9998 & 0 & -0.01745 & -0.1258\\
0.01745 & 0 & 0.9998 & 0.4716\\
0 & 0 & 0 & 1
\end{bmatrix}
\end{split}
\end{equation}

\begin{equation}
\begin{split}
^{0}T_{5}&=^{0}T_{4}\cdot^{4}T_{5}
&=\begin{bmatrix}
0 & 0 & -1 & -0.1333\\
0.9998 & 0.01745 & 0 & -0.1276\\
0.01745 & -0.9998 & 0 & 0.5713\\
0 & 0 & 0 & 1
\end{bmatrix}
\end{split}
\end{equation}

\subsubsection{Frame Transformation Matrices of Joint Position B}
Obtain the DH table of the robot at joint position B by converting specified joint angles from degrees to radians:
\begin{table}[!ht]
    \caption{DH Parameters of UR5e Robot at position B}
    \begin{center}
    \begin{tabular}{||c | c | c | c | c||} 
     \hline
     Link & $\theta$ (rad) & a (m) & d (m) & $\alpha$ (rad)\\
     \hline\hline
     1 & $-\frac{\pi}{2}$ & 0 & 0.1625 & $\frac{\pi}{2}$ \\ [1ex] 
     \hline
     2 & $-\frac{\pi}{3}$ & -0.425 & 0 & 0 \\ [1ex] 
     \hline
     3 & $\frac{\pi}{2}$ & -0.3922 & 0 & 0 \\ [1ex] 
     \hline
     4 & 0 & 0 & 0.1333 & $\frac{\pi}{2}$ \\ [1ex] 
     \hline
     5 & $\frac{\pi}{2}$ & 0 & 0.0997 & $-\frac{\pi}{2}$ \\ [1ex] 
     \hline
     6 & 0 & 0 & 0.0996 & 0 \\ [1ex] 
     \hline
    \end{tabular}
    \end{center}
\end{table}

Now, the consecutive joint frame transformations can be obtained by substitution of the above values: 
\begin{equation}
\begin{split}
^{0}T_{1}&=\begin{bmatrix}
\cos(-\frac{\pi}{2}) & -\sin(-\frac{\pi}{2})\cos(\frac{\pi}{2}) & \sin(-\frac{\pi}{2})\sin(\frac{\pi}{2}) & 0\cos(-\frac{\pi}{2})\\
\sin(-\frac{\pi}{2}) & \cos(-\frac{\pi}{2})\sin(\frac{\pi}{2}) & -\cos(-\frac{\pi}{2})\sin(\frac{\pi}{2}) & 0\sin(-\frac{\pi}{2})\\
0 & \sin(\frac{\pi}{2}) & \cos(\frac{\pi}{2}) & 0.1625\\
0 & 0 & 0 & 1
\end{bmatrix}\\
&=\begin{bmatrix}
0 & 0 & -1 & 0\\
-1 & 0 & 0 & 0\\
0 & 1 & 0 & 0.1625\\
0 & 0 & 0 & 1
\end{bmatrix}
\end{split}
\end{equation}

\begin{equation}
\begin{split}
^{1}T_{2}&=\begin{bmatrix}
0.5 & 0.8660 & 0 & -0.2125\\
-0.8660 & 0.5 & 0 & 0.3681\\
0 & 0 & 1 & 0\\
0 & 0 & 0 & 1
\end{bmatrix}
\end{split}
\end{equation}

\begin{equation}
\begin{split}
^{2}T_{3}&=\begin{bmatrix}
0 & -1 & 0 & 0\\
1 & 0 & 0 & -0.3922\\
0 & 0 & 1 & 0\\
0 & 0 & 0 & 1
\end{bmatrix}
\end{split}
\end{equation}

\begin{equation}
\begin{split}
^{3}T_{4}&=\begin{bmatrix}
1 & 0 & 0 & 0\\
0 & 0 & -1 & 0\\
0 & 1 & 0 & 0.1333\\
0 & 0 & 0 & 1
\end{bmatrix}
\end{split}
\end{equation}

\begin{equation}
\begin{split}
^{4}T_{5}&=\begin{bmatrix}
0 & 0 & -1 & 0\\
1 & 0 & 0 & 0\\
0 & -1 & 0 & 0.0997\\
0 & 0 & 0 & 1
\end{bmatrix}
\end{split}
\end{equation}

\begin{equation}
\begin{split}
^{5}T_{6}&=\begin{bmatrix}
1 & 0 & 0 & 0\\
0 & 1 & 0 & 0\\
0 & 0 & 1 & 0.0996\\
0 & 0 & 0 & 1
\end{bmatrix}
\end{split}
\end{equation}

With the consecutive joint frame transformations, the transformations between each joint and the base frame are easily obtained:

\begin{equation}
\begin{split}
^{0}T_{2}&=^{0}T_{1}\cdot^{1}T_{2}
&=\begin{bmatrix}
0 & 0 & -1 & 0\\
-0.5 & -0.8660 & 0 & 0.2125\\
-0.8660 & 0.5 & 0 & 0.5306\\
0 & 0 & 0 & 1
\end{bmatrix}
\end{split}
\end{equation}

\begin{equation}
\begin{split}
^{0}T_{3}&=^{0}T_{2}\cdot^{2}T_{3}
&=\begin{bmatrix}
0 & 0 & -1 & 0\\
-0.8660 & 0.5 & 0 & 0.5522\\
0.5 & 0.8660 & 0 & 0.3345\\
0 & 0 & 0 & 1
\end{bmatrix}
\end{split}
\end{equation}

\begin{equation}
\begin{split}
^{0}T_{4}&=^{0}T_{3}\cdot^{3}T_{4}
&=\begin{bmatrix}
0 & -1 & 0 & -0.1333\\
-0.8660 & 0 & -0.5 & 0.5522\\
0.5 & 0 & -0.8660 & 0.3345\\
0 & 0 & 0 & 1
\end{bmatrix}
\end{split}
\end{equation}

\begin{equation}
\begin{split}
^{0}T_{5}&=^{0}T_{4}\cdot^{4}T_{5}
&=\begin{bmatrix}
-1 & 0 & 0 & -0.1333\\
0 & 0.5 & 0.8660 & 0.5023\\
0 & 0.8660 & -0.5 & 0.2481\\
0 & 0 & 0 & 1
\end{bmatrix}
\end{split}
\end{equation}

\subsection{Calculation of Forward Kinematic Solutions}
\\subsubsection{Forward Kinematic Solution for Joint Position A}
\begin{equation}
\begin{split}
^{0}T_{6}&=^{0}T_{5}\cdot^{5}T_{6}
&=\begin{bmatrix}
0 & 0 & -1 & -0.2329\\
0.9998 & 0.01745 & 0 & -0.1276\\
0.01745 & -0.9998 & 0 & 0.5713\\
0 & 0 & 0 & 1
\end{bmatrix}
\end{split}
\end{equation}

\\subsubsection{Forward Kinematic Solution for Joint Position B}
\begin{equation}
\begin{split}
^{0}T_{2}&=^{0}T_{5}\cdot^{5}T_{6}
&=\begin{bmatrix}
-1 & 0 & 0 & -0.1333\\
0 & 0.5 & 0.8660 & 0.5886\\
0 & 0.8660 & -0.5 & 0.1983\\
0 & 0 & 0 & 1
\end{bmatrix}
\end{split}
\end{equation}

\subsection{Verification of Solutions via MATLAB}
\\subsubsection{Forward Kinematic Solution for Joint Position A}
\begin{equation}
\begin{split}
^{0}T_{6}&=\begin{bmatrix}
0 & 0 & -1 & -0.2329\\
0.9998 & 0.01745 & 0 & -0.1276\\
0.01745 & -0.9998 & 0 & 0.5713\\
0 & 0 & 0 & 1
\end{bmatrix}
\end{split}
\end{equation}

\\subsubsection{Forward Kinematic Solution for Joint Position B}
\begin{equation}
\begin{split}
^{0}T_{2}&=\begin{bmatrix}
-1 & 0 & 0 & -0.1333\\
0 & 0.5 & 0.8660 & 0.5886\\
0 & 0.8660 & -0.5 & 0.1983\\
0 & 0 & 0 & 1
\end{bmatrix}
\end{split}
\end{equation}

\subsection{Verification of Solutions via URSim}

\begin{enumerate}
    \item Test
    \item Test
\end{enumerate}

\section{Part B: Calculation of Jacobian}
\subsection{Calculation of Jacobians}

\subsection{Calculation of End Effector Velocities}

\section{Part C: Trajectory Generation}
\subsection{Square Trajectory via Tool-Space Movements}

\subsection{Square Trajectory via Joint-Space Movements}

\subsection{Comparison of Simulated Tool and Joint-Space Trajectories}

\subsection{Comparison of Real-World Tool and Joint-Space Trajectories}

\subsection{Discussion on the Trajectory Generation Method of the UR5e Robot}

\subsection{Applications for Movement in Tool and Joint-Space}

\section{Part D: Singularities}
\subsection{Path Plan Between Set Poses in Joint-Space}

\subsection{Path Plan Between Set Poses in Tool-Space}

\subsection{Discussion of Suitability of Planned Paths}

\subsection{Discussion of the Types and Effects of Robotic Joint Singularities}
A singularity is a state of a robot's joint configuration parameters which results in the loss of one or more degrees of freedom (DOFs) in its motion.
In terms of the Jacobian, a singularity is said to be reached if the determinant of the Jacobian for a given joint configuration is zero. Effectively,
this means that not all of the columns of the Jacobian matrix are linearly independent of one another, which explains why a singularity state results in the loss
of a degree of freedom, since a change in one joint parameter will be equivalent to a change in another. The three types of singularities which are relevant to
6-DOF robots such as the UR5e are: wrist singularities, shoulder singularities and elbow singularities.

A wrist singularity is reached when the actuation axes of joints 4 and 6 (the first and last wrist joints respectively) are parallel.

A shoulder singularity is reached when the robot's wrist center is aligned with the actuation axis of the first, or base, joint. In this singularity state, rotation of the robot's wrist
in one direction will be equivalent to rotation of the base joint.

An elbow singularity is reached when the actuation axes of joints 2 and 3 are aligned with the robot's wrist center. Effectively, the robot has extended itself to the limit
of its workspace in this configuration, meaning that it is not possible for it to extend further outwards in a radial direction from its base frame.

\end{document}