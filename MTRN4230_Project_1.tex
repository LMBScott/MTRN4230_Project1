\documentclass[fleqn]{article}
\usepackage[a4paper, total={6in, 10in}]{geometry}
\usepackage{graphicx}
\usepackage{amsmath}
\graphicspath{ {./images/} }

\title{MTRN4230 Project 1}
\author{Lachlan Scott | z5207471}
\date{July 10, 2022}

\begin{document}
\maketitle

\section{Provided Data}
\subsection{Defined positions:}
A and B are two sets of joint angle positions ($\theta_1$ to $\theta_6$) of a UR5e robot:
\begin{itemize} 
    \item Joint Variables/Position A: [ -90, -173, 132, 220, 0, 0] (degrees)
    \item Joint Variables/Position B: [-90, -60, 90, 0, 90, 0] (degrees)
\end{itemize}
\subsection{Denavit-Hartenburg Parameter Table of UR5e Robot:}

\begin{table}[!ht]
    \caption{DH Parameters of UR5e Robot}
    \begin{center}
    \begin{tabular}{||c | c | c | c | c||} 
     \hline
     Link & $\theta$ (rad) & a (m) & d (m) & $\alpha$ (rad)\\
     \hline\hline
     1 & 0 & 0 & 0.1625 & $\frac{\pi}{2}$ \\ [1ex] 
     \hline
     2 & 0 & -0.425 & 0 & 0 \\ [1ex] 
     \hline
     3 & 0 & -0.3922 & 0 & 0 \\ [1ex] 
     \hline
     4 & 0 & 0 & 0.1333 & $\frac{\pi}{2}$ \\ [1ex] 
     \hline
     5 & 0 & 0 & 0.0997 & $-\frac{\pi}{2}$ \\ [1ex] 
     \hline
     6 & 0 & 0 & 0.0996 & 0 \\ [1ex] 
     \hline
    \end{tabular}
    \end{center}
\end{table}

\section{Part A: Kinematic Modelling}
\subsection{Construction of a Model of the UR5e Robot}

\subsection{Calculation of Frame Transformation Matrices}
\subsubsection{Useful Definitions}
Translation matrix describing a translation from frame A to frame B:
\begin{equation}
\begin{split}
^{A}p_{B}=\begin{bmatrix}
^{A}x_{B}\\
^{A}y_{B}\\
^{A}z_{B}
\end{bmatrix}
\end{split}
\end{equation}
Rotation matrix describing a rotation from frame A to frame B:
\begin{equation}
\begin{split}
^{A}R_{B}=\begin{bmatrix}
x_b\cdot x_a & y_b\cdot x_a & z_b\cdot x_a\\
x_b\cdot y_a & y_b\cdot y_a & z_b\cdot y_a\\
x_b\cdot z_a & y_b\cdot z_a & z_b\cdot z_a
\end{bmatrix}
\end{split}
\end{equation}
Transformation between consecutive joints using DH convention:
\begin{equation}
\begin{split}
^{i-1}T_{i}&=R_{i-1}(\theta_i)\cdot Q_{i-1}(d_i)\cdot Q_i(a_i)\cdot R_i(\alpha_i)\\
&=\begin{bmatrix}
\cos(\theta_i) & -\sin(\theta_i)\cos(\alpha_i) & \sin(\theta_i)\sin(\alpha_i) & a_i\cos(\theta_i)\\
\sin(\theta_i) & \cos(\theta_i)\sin(\alpha_i) & -\cos(\theta_i)\sin(\alpha_i) & a_i\sin(\theta_i)\\
0 & \sin(\alpha_i) & \cos(\alpha_i) & d_i\\
0 & 0 & 0 & 1
\end{bmatrix}
\end{split}
\end{equation}
\subsubsection{Frame Transformation Matrices of Joint Position A}
Obtain the DH table of the robot at joint position A by converting specified joint angles from degrees to radians:
\begin{table}[!ht]
    \caption{DH Parameters of UR5e Robot at position A}
    \begin{center}
    \begin{tabular}{||c | c | c | c | c||} 
     \hline
     Link & $\theta$ (rad) & a (m) & d (m) & $\alpha$ (rad)\\
     \hline\hline
     1 & $-\frac{\pi}{2}$ & 0 & 0.1625 & $\frac{\pi}{2}$ \\ [1ex] 
     \hline
     2 & -3.019 & -0.425 & 0 & 0 \\ [1ex] 
     \hline
     3 & 2.304 & -0.3922 & 0 & 0 \\ [1ex] 
     \hline
     4 & 3.840 & 0 & 0.1333 & $\frac{\pi}{2}$ \\ [1ex] 
     \hline
     5 & 0 & 0 & 0.0997 & $-\frac{\pi}{2}$ \\ [1ex] 
     \hline
     6 & 0 & 0 & 0.0996 & 0 \\ [1ex] 
     \hline
    \end{tabular}
    \end{center}
\end{table}

Now, the required frame transformations can be obtained by substitution of the above values: 
\begin{equation}
\begin{split}
^{0}T_{1}&=\begin{bmatrix}
\cos(-\frac{\pi}{2}) & -\sin(-\frac{\pi}{2})\cos(\frac{\pi}{2}) & \sin(-\frac{\pi}{2})\sin(\frac{\pi}{2}) & 0\cos(-\frac{\pi}{2})\\
\sin(-\frac{\pi}{2}) & \cos(-\frac{\pi}{2})\sin(\frac{\pi}{2}) & -\cos(-\frac{\pi}{2})\sin(\frac{\pi}{2}) & 0\sin(-\frac{\pi}{2})\\
0 & \sin(\frac{\pi}{2}) & \cos(\frac{\pi}{2}) & 0.1625\\
0 & 0 & 0 & 1
\end{bmatrix}\\
&=\begin{bmatrix}
0 & 0 & -1 & 0\\
-1 & 0 & 0 & 0\\
0 & 1 & 0 & 0.1625\\
0 & 0 & 0 & 1
\end{bmatrix}
\end{split}
\end{equation}

\begin{equation}
\begin{split}
^{1}T_{2}&=\begin{bmatrix}
-0.9925 & 0.1223 & 0 & 0.4218\\
-0.1223 & -0.9925 & 0 & 0.0520\\
0 & 0 & 1 & 0\\
0 & 0 & 0 & 1
\end{bmatrix}
\end{split}
\end{equation}

\begin{equation}
\begin{split}
^{2}T_{3}&=\begin{bmatrix}
-0.6693 & -0.7430 & 0 & 0.2625\\
0.7430 & -0.6693 & 0 & -0.2914\\
0 & 0 & 1 & 0\\
0 & 0 & 0 & 1
\end{bmatrix}
\end{split}
\end{equation}

\begin{equation}
\begin{split}
^{3}T_{4}&=\begin{bmatrix}
-0.7659 & 0 & -0.6430 & 0\\
-0.6430 & 0 & 0.7659 & 0\\
0 & 1 & 0 & 0.1333\\
0 & 0 & 0 & 1
\end{bmatrix}
\end{split}
\end{equation}

\begin{equation}
\begin{split}
^{4}T_{5}&=\begin{bmatrix}
1 & 0 & 0 & 0\\
0 & 0 & 1 & 0\\
0 & -1 & 0 & 0.0997\\
0 & 0 & 0 & 1
\end{bmatrix}
\end{split}
\end{equation}

\begin{equation}
\begin{split}
^{5}T_{6}&=\begin{bmatrix}
1 & 0 & 0 & 0\\
0 & 1 & 0 & 0\\
0 & 0 & 1 & 0.0996\\
0 & 0 & 0 & 1
\end{bmatrix}
\end{split}
\end{equation}

\subsection{Calculation of Forward Kinematic Solutions}

\subsection{Verification of Solutions via MATLAB}

\subsection{Verification of Solutions via URSim}

\begin{enumerate}
    \item Test
    \item Test
\end{enumerate}

\section{Part B: Calculation of Jacobian}
\subsection{Calculation of Jacobians}

\subsection{Calculation of End Effector Velocities}

\section{Part C: Trajectory Generation}
\subsection{Square Trajectory via Tool-Space Movements}

\subsection{Square Trajectory via Joint-Space Movements}

\subsection{Comparison of Simulated Tool and Joint-Space Trajectories}

\subsection{Comparison of Real-World Tool and Joint-Space Trajectories}

\subsection{Discussion on the Trajectory Generation Method of the UR5e Robot}

\subsection{Applications for Movement in Tool and Joint-Space}

\section{Part D: Singularities}
\subsection{Path Plan Between Set Poses in Joint-Space}

\subsection{Path Plan Between Set Poses in Tool-Space}

\subsection{Discussion of Suitability of Planned Paths}

\subsection{Discussion of the Types and Effects of Robotic Joint Singularities}

\end{document}